\chapter{RT"=Notation}
\label{chapter:Anhang-RtNotation}

{\scriptsize\begin{verbatim}
LABEL       BEFEHL #KOMMENTAR
    #übergeordneter Kommentar

START [PAKETFELDLÄNGE als AT]
            FILLEDTMP = AT / 32 #Länge in Speicherzellen umwandeln
            TABELLENANFANG = ACCU+FILLEDTMP
            #erste Speicherzelle von Paketfeld in BUFFER_ONE einlesen
            MAR = ZAEHLER = ACCU
            BUFFER_ONE = M[MAR]
PAKETFELD
            ZAEHLER++
            #prüfen, ob Paketfeld noch nicht zu Ende, sonst nach END gehen
            IF TABELLENANFANG - ZAEHLER == 0
                1: GOTO END
            #BUFFER_TWO mit nächster Speicherzelle füllen
            MAR = ZAEHLER
            BUFFER_TWO = M[MAR]
            FILLED = 32 #BUFFER_TWO ist jetzt mit 32Bit gefüllt
BIGREPEAT   #Repeat bis BUFFER_TWO leer ist
            #erste 4 Bits von BUFFER_ONE in ACCU speichern
            ACCU = BUFFER_ONE * 2^28
            #wenn kein Headeranfang gehe zu DATA
            IF ACCU - STARTSEQ == 0
                0: GOTO DATA
    #jump 32 Bit zur Kanalnr
            #check ob BUFFER_TWO voll ist, dann kann man BUFFER_ONE komplett ersetzen
            IF FILLED - 32 == 0
                0: GOTO JUMP1
            BUFFER_ONE = BUFFER_TWO
            MAR = ZAEHLER = ZAEHLER + 1
            BUFFER_TWO = M[MAR]
            GOTO KANALSP
JUMP1
            FILLEDTMP = FILLED
WHILE1
            #BUFFER_ONE solange füllen, bis BUFFER_TWO leer (WHILE2)
            IF FILLED == 0
                1:GOTO WHILE2
            BUFFER_ONE /= 2
            ACCU = BUFFER_TWO * 2^31
            BUFFER_ONE += ACCU
            BUFFER_TWO /= 2
            FILLED--
            GOTO WHILE1
WHILE2
            #BUFFER_TWO nachfüllen
            MAR = ZAEHLER = ZAEHLER + 1
            BUFFER_TWO = M[MAR]
            FILLED=32
REPEAT
            BUFFER_ONE /= 2
            ACCU = BUFFER_TWO * 2^31
            BUFFER_ONE += ACCU
            BUFFER_TWO /= 2
            FILLED--;
            #BUFFER_ONE nachfüllen bis 32Bit von BUFFER_TWO drin sind
            IF FILLEDTMP - FILLED == 0
                0: GOTO REPEAT
KANALSP
    #speichere KANALNR (16 Bit)
            ACCU = BUFFER_ONE * 2^16
            KANALNR = ACCU / 2^16
    #jump 16 Bit
            FILLEDTMP = 16
REPEAT2
            #wenn BUFFER_TWO leer, dann auffüllen
            IF FILLED==0
                0: GOTO 3WEITER
            MAR = ZAEHLER = ZAEHLER + 1
            BUFFER_TWO = M[MAR]
            FILLED = 32
3WEITER
            #repeat2 weitermachen
            BUFFER_ONE /= 2
            ACCU = BUFFER_TWO * 2^31
            BUFFER_ONE += ACCU
            BUFFER_TWO /= 2
            FILLED--
            FILLEDTMP--
            #16 mal durchlaufen, also FILLEDTMP=0. JUMP16 fertig
            IF FILLEDTMP == 0;
                0: GOTO REPEAT2
    #jump 32 Bit (siehe oben)
            GOTO HINTERDATA
DATA
    #Datenteil zählen
            #Bit Anzahl abspeichern und nächstes bit laden
            MAR = TABELLENANFANG + KANALNR
            M[MAR]++
            BUFFER_ONE /= 2
            ACCU = BUFFER_TWO * 2^31
            BUFFER_ONE += ACCU
            BUFFER_TWO /= 2
            FILLED--
HINTERDATA
            IF FILLED==0
                0: GOTO BIGREPEAT
                1: GOTO PAKETFELD
END
\end{verbatim}}