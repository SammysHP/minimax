\chapter{Konfiguration}
\label{chapter:Anhang-Konfiguration}

{\footnotesize\linespread{0.55}\linenumbers\begin{verbatim}START ERWEITERUNG

  START REGISTER
  # hier sind alle neuen Register einzutragen
    %
    Name=TABELLENANFANG
    Nummer=0
    Breite=32
    Steuername=TABELLENANFANG.W
    %%
    
    %
    Name=FILLEDTMP
    Nummer=0
    Breite=32
    Steuername=FILLEDTMP.W
    %%

    %
    Name=KANALNR
    Nummer=0
    Breite=32
    Steuername=KANALNR.W
    %%

    %
    Name=FILLED
    Nummer=0
    Breite=32
    Steuername=FILLED.W
    %%

    %
    Name=BUFFER_ONE
    Nummer=0
    Breite=32
    Steuername=BUFFER_ONE.W
    %%

    %
    Name=BUFFER_TWO
    Nummer=0
    Breite=32
    Steuername=BUFFER_TWO.W
    %%

  ENDE REGISTER
  
  
  START MULTIPLEXER
  # hier gibt man für die zu ändernden Multiplexer alle Angaben an dabei muß der Name
  # und die Nummer mit dem entsprechenden Multiplexers der Basismaschine übereinstimmen
  
    %
    Steuername=ALUSel.A
    Nummer=0
    Eingaenge=13
    %%
    
    %
    Steuername=ALUSel.B
    Nummer=0
    Eingaenge=10
    %%
  
  
  ENDE MULTIPLEXER
  
  START ALU

    %    
    Funktionen=ADD,SUB2,TRANS.A,TRANS.B,MUL,SR.B,AND
    %%    
     
  ENDE ALU
    
  START CU
    # nichts zu tun  
  ENDE CU
  
  START SIGN
  #bei Bedarf einfügen
    
    #%  
    #Name=sign ext.
    #Nummer=1
    #Breite=24
    %%  
  
  ENDE SIGN
    
  START SPEICHER
   # nichts zu tun  
  ENDE SPEICHER
  
    
  START VERBINDUNGEN
  # hier sind alle(!) Verbindungen einzutragen
    #Ausgang,Nummer,Port;Eingang,Nummer,Port;
    
    #START SPEICHER  
      MEM,0,DO;MDR.Sel,0,1
    #ENDE SPEICHER 
    
    #START REGISTER
      MAR,0,0;MEM,0,A
      MDR,0,0;MEM,0,DI;ALUSel.B,0,1
      IR,0,0;CU,1,0;sign ext.,0,0
      ACCU,0,0;ALUSel.A,0,0;ALUSel.B,0,0
      KANALNR,0,0;ALUSel.A,0,9
      TABELLENANFANG,0,0;ALUSel.B,0,3
      FILLED,0,0;ALUSel.A,0,7;ALUSel.B,0,7
      FILLEDTMP,0,0;ALUSel.B,0,6
      PC,0,0;ALUSel.A,0,8;ALUSel.B,0,8
      BUFFER_ONE,0,0;ALUSel.B,0,4
      BUFFER_TWO,0,0;ALUSel.B,0,5
    #ENDE REGISTER
    
    #START SIGNEXT
      sign ext.,0,0;ALUSel.B,0,2
    #ENDE SIGNEXT
    
    #START MULTIPLEXER
      MDR.Sel,0,0;MDR,0,0
      ALUSel.A,0,0;ALU,0,A
      ALUSel.B,0,0;ALU,0,B
    #ENDE MULTIPLEXER
        
    #START ALU
      ALU,0,0;TABELLENANFANG,0,0;FILLEDTMP,0,0;KANALNR,0,0;FILLED,0,0;
        BUFFER_ONE,0,0;BUFFER_TWO,0,0;ACCU,0,0;PC,0,0;MDR.Sel,0,0;IR,0,0;MAR,0,0
      ALU==0,ALU,0;CU,0,0
    #ENDE ALU

    #START KONSTANTEN
    # 2er-Potenzen: 2^31
      &&,0,0;ALUSel.A,0,1
      &&,1,0;ALUSel.A,0,2
      &&,32,0;ALUSel.A,0,3
      &&,2^16,0;ALUSel.A,0,4
      &&,2^28,0;ALUSel.A,0,5
      &&,2^31,0;ALUSel.A,0,6
      #e
      &&,14,0;ALUSel.B,0,9
      &&,65535,0;ALUSel.A,0,11
      &&,16,0;ALUSel.A,0,10
      &&,2,0;ALUSel.A,0,12
    #ENDE KONSTANTEN
  
  ENDE VERBINDUNGEN
  
ENDE ERWEITERUNG
\end{verbatim}}