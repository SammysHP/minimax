\chapter{Benchmark und Bewertung}
\label{chapter:Dokumentation-BenchmarkBewertung}

\section{Berechnung}
\label{section:Dokumentation-BenchmarkBewertung-Berechnung}

Für die Berechnung der Laufzeit und für die Bewertung des Algorithmus gelten folgende Vorschriften:

\subsection{Variablen}
\label{subsection:Dokumentation-BenchmarkBewertung-Berechnung-Variablen}

\begin{description}
    \item[reg] Anzahl der ergänzten Minimaxregister
    \item[se] Anzahl der ergänzten Sign"=Extension"=Units (0 oder 1)
    \item[const] Anzahl der ergänzten Konstanten (eine "`0"' und eine "`1"' frei)
    \item[alu\_add] Penaltysumme für alle ergänzten ALU"=Befehle
    \item[alu\_use] Penaltysumme der \emph{verwendeten} ALU"=Befehle im Programm
\end{description}

\begin{center}
    \begin{tabular}{|c||c|c|c|c|c|c|c|}
        \hline
        extra ALU"=Op. & SUB1 & INC/DEC & S.L, S.R & AND, OR, NOT & XOR & DIV & Custom \\ 
        \hline
        Penalty (1..20) & 1 & 4 & 5 & 6 & 8 & 10 & bis zu 20 \\
        \hline
    \end{tabular}
\end{center}

\subsubsection{Variablenwerte}
\label{subsubsection:Dokumentation-BenchmarkBewertung-Berechnung-Variablen-Variablenwerte}

\todo{Werte bestimmen, einsetzen und berechnen}

Im Folgenden sind die Werte für den verwendeten Algoritmus angegeben:

\begin{align*}
    reg      &= 7 \\
    se       &= 0 \\
    const    &= 6 \\
    alu\_add &= 10 + 10 = 20 \\
    alu\_use &= 1 + 1 + 10 + 10 + 1 = 23
\end{align*}

\subsection{Laufzeit}
\label{subsection:Dokumentation-BenchmarkBewertung-Berechnung-Laufzeit}

Die Laufzeit berechnet sich nach folgender Formel:

\begin{align*}
    t_{bewertet} &= t_{bench} \cdot (1 + 0.1 \cdot reg + 0.15 \cdot se + 0.015 \cdot alu\_add + 0.05 \cdot const) \\
                 &= t_{bench} \cdot (1 + 0.1 \cdot 7 + 0.15 \cdot 0 + 0.015 \cdot 20 + 0.05 \cdot 6) \\
                 &= t_{bench} \cdot 2.3
\end{align*}

Zur Bestimmung von $t_{bench}$ wurden 3 Speicherabbilder zur Verfügung gestellt. \todo{ausformulieren, berechnen}

\subsection{Länge}
\label{subsection:Dokumentation-BenchmarkBewertung-Berechnung-Laenge}

Die Länge berechnet sich nach folgender Formel:

\begin{align*}
    n_{bewertet} &= n_{Algorithmus} + 5 \cdot reg + 10 \cdot se + 3 \cdot alu\_use + 5 \cdot const \\
                 &= 100 + 5 \cdot 7 + 10 \cdot 0 + 3 \cdot 23 + 5 \cdot 6 \\
                 &= 234
\end{align*}

\section{Abschließende Bewertung}
\label{section:Dokumentation-BenchmarkBewertung-Bewertung}

Algorithmus ist für normale Anwendung geeignet bla blubb... \todo{Abschließende Bewertung}
