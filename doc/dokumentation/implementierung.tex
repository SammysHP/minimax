\chapter{Implementierung}
\label{chapter:Dokumentation-Implementierung}

\section{Hardwareerweiterung}
\label{section:Dokumentation-Implementierung-Hardwareerweiterung}

\subsection{Register}
\label{subsection:Dokumentation-Implementierung-Hardwareerweiterung-Register}

\begin{description}
    \item[BUFFER\_ONE] immer voller Buffer, bitgenau
    \item[BUFFER\_TWO] füllt \texttt{BUFFER\_ONE}, wird mit Speicherzellen gefüllt
    \item[FILLED] Füllstand von \texttt{BUFFER\_TWO}
    \item[FILLEDTMP] Zwischenspeicher
    \item[ZAHLER] aktuelle Adresse vom Paketfeld
    \item[TABELLENANFANG]
    \item[KANALNR]
\end{description}

\subsection{ALU"=Operationen}
\label{subsection:Dokumentation-Implementierung-Hardwareerweiterung-AluOps}

\todo{ggf. erweitern}

\begin{description}
    \item[MUL] $ALUresult \gets A \cdot B$
    \item[DIV.A] $ALUresult \gets B / A$
    \item[AND] $ALUresult \gets A & B$
\end{description}

\subsection{Konstanten}
\label{subsection:Dokumentation-Implementierung-Hardwareerweiterung-Konstanten}

\begin{itemize}
    \item $(14)_{10}$
    \item $(2^{31})_{10}$
    \item $(2^{28})_{10}$
    \item $(2^{16})_{10}$
    \item $(2^5)_{10}$
    \item $(2)_{10}$
    \item $(16)_{10}$
    \item $(65535)_{10}$
\end{itemize}

\subsection{Konfigurationsdatei}
\label{subsection:Dokumentation-Implementierung-Hardwareerweiterung-Konfigurationsdatei}

Mit Hilfe der bereitgestellten Beispiele konnte eine Erweiterungsdatei für die die oben genannten Erweiterungen erstellt werden. Diese kann in den Simulator geladen werden (siehe \autoref{chapter:Dokumentation-Simulation}) um die Hardware zu erweitern.

Zwar besitzt der Simulator eine grafische Oberfläche zum Erstellen dieser Konfigurationsdatei, jedoch war die Bedienung unschlüssig und ein direktes Verfassen der Konfigurationsdatei per Hand um ein vielfaches schneller.

Die Erweiterungskonfiguration befindet sich in \autoref{chapter:Anhang-Konfiguration}.

\subsection{Maschinencodierung}
\label{subsection:Dokumentation-Implementierung-Hardwareerweiterung-Maschinencodierung}

\textbf{ALU"=Operationen:}

\begin{tabular}{lc}
    Operation & Code \\
    \hline
    ADD       & 000 \\
    SUB.B     & 001 \\
    TRANS.A   & 010 \\
    TRANS.B   & 011 \\
    MUL       & 100 \\
    DIV.A     & 101 \\
    AND       & 110 \\
\end{tabular}

\textbf{ALUSel.A:}

\begin{tabular}{lcc}
    Quelle               & Code & Eingang \\
    \hline
    ACCU                 & 0000 &  0 \\
    PC                   & 0001 &  1 \\
    1                    & 0010 &  2 \\
    (32)_{10}            & 0011 &  3 \\
    (2^{16})_{10}        & 0100 &  4 \\
    (2^{28})_{10}        & 0101 &  5 \\
    (2^{31})_{10}        & 0110 &  6 \\
    FILLED               & 0111 &  7 \\
    ZAEHLER              & 1000 &  8 \\
    KANALNR              & 1001 &  9 \\
    (16)_{10}            & 1010 & 10 \\
    (65535)_{10}         & 1011 & 11 \\
    (2)_{10}             & 1100 & 12 \\
\end{tabular}

\textbf{ALUSel.B:}

\begin{tabular}{lcc}
    Quelle        & Code & Eingang \\
    \hline
    ACCU          & 0000 &  0 \\
    MDR           & 0001 &  1 \\
    AT(signext)   & 0010 &  2 \\
    TABELLENANF   & 0011 &  3 \\
    BUFFER\_ONE   & 0100 &  4 \\
    BUFFER\_TWO   & 0101 &  5 \\
    FILLEDTMP     & 0110 &  6 \\
    FILLED        & 0111 &  7 \\
    ZAEHLER       & 1000 &  8 \\
    (14)_{10}     & 1001 &  9 \\
\end{tabular}

\section{Algorithmus}
\label{section:Dokumentation-Implementierung-Algorithmus}

\todo{Algorithmus erklären}

\section{Steuertabelle}
\label{section:Dokumentation-Implementierung-Steuertabelle}

Da der Algorithmus bereits in der maschinennahen RT"=Notation verfasst wurde, konnte die Steuertabelle aus diesem mit Hilfe der Maschinencodierung (s.o.) direkt abgeleitet werden. Die Steuertabelle befindet sich in \autoref{chapter:Anhang-Steuertabelle}.
