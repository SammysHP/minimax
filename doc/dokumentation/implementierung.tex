\chapter{Implementierung}
\label{chapter:Dokumentation-Implementierung}

\section{Hardwareerweiterung}
\label{section:Dokumentation-Implementierung-Hardwareerweiterung}

\subsection{Register}
\label{subsection:Dokumentation-Implementierung-Hardwareerweiterung-Register}

\begin{description}
    \item[BUFFER\_ONE] immer voller Buffer, bitgenau
    \item[BUFFER\_TWO] füllt \texttt{BUFFER\_ONE}, wird mit Speicherzellen gefüllt
    \item[FILLED] Füllstand von \texttt{BUFFER\_TWO}
    \item[FILLEDTMP] Zwischenspeicher
    \item[ZAHLER] aktuelle Adresse vom Paketfeld
    \item[TABELLENANFANG]
    \item[KANALNR]
\end{description}

\subsection{ALU"=Operationen}
\label{subsection:Dokumentation-Implementierung-Hardwareerweiterung-AluOps}

\todo{ggf. erweitern}

\begin{description}
    \item[MUL] $ALUresult \gets A \cdot B$
    \item[DIV.A] $ALUresult \gets B / A$
\end{description}

\subsection{Konstanten}
\label{subsection:Dokumentation-Implementierung-Hardwareerweiterung-Konstanten}

\begin{itemize}
    \item $(e0000000)_{16}$
    \item $(2^{31})_{10}$
    \item $(2^{28})_{10}$
    \item $(2^{16})_{10}$
    \item $(2^5)_{10}$
    \item $(2)_{10}$
\end{itemize}

\subsection{Konfigurationsdatei}
\label{subsection:Dokumentation-Implementierung-Hardwareerweiterung-Konfigurationsdatei}

Mit Hilfe der bereitgestellten Beispiele konnte eine Erweiterungsdatei für die die oben genannten Erweiterungen erstellt werden. Diese kann in den Simulator geladen werden (siehe \autoref{chapter:Dokumentation-Simulation}) um die Hardware zu erweitern.

Zwar besitzt der Simulator eine grafische Oberfläche zum Erstellen dieser Konfigurationsdatei, jedoch war die Bedienung unschlüssig und ein direktes Verfassen der Konfigurationsdatei um ein vielfaches schneller.

Die Erweiterungskonfiguration befindet sich in \autoref{chapter:Anhang-Konfiguration}.

\section{Algorithmus}
\label{section:Dokumentation-Implementierung-Algorithmus}

Der Algorithmus wurde direkt in der maschinennahen RT"=Notation verfasst. Dadurch fallen zum einen maschinenspezifische Vor"= und Nachteile direkt auf, zum anderen ist die Umwandlung in die Steuertabelle einfacher. Der fertige Algorithmus befindet sich in \autoref{chapter:Anhang-RtNotation}.

\section{Steuertabelle}
\label{section:Dokumentation-Implementierung-Steuertabelle}

Die Steuertabelle ist eine 1:1 Übertragung der RT"=Notation in Steueranweisungen für die Hardware. Sie befindet sich in \autoref{chapter:Anhang-Steuertabelle}.
