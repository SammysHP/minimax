\chapter{Implementierung}
\label{chapter:Dokumentation-Implementierung}

\section{Hardwareerweiterung}
\label{section:Dokumentation-Implementierung-Hardwareerweiterung}

\subsection{Register}
\label{subsection:Dokumentation-Implementierung-Hardwareerweiterung-Register}

\begin{description}
    \item[BUFFER\_ONE] immer voller Buffer, bitgenau
    \item[BUFFER\_TWO] füllt \texttt{BUFFER\_ONE}, wird mit Speicherzellen gefüllt
    \item[FILLED] Füllstand von \texttt{BUFFER\_TWO}
    \item[FILLEDTMP] Zwischenspeicher
    \item[ZAHLER] aktuelle Adresse vom Paketfeld
    \item[TABELLENANFANG]
    \item[KANALNR]
\end{description}

\subsection{ALU"=Operationen}
\label{subsection:Dokumentation-Implementierung-Hardwareerweiterung-AluOps}

\todo{ggf. erweitern}

\begin{description}
    \item[MUL] $ALUresult \gets A \cdot B$
    \item[DIV] $ALUresult \gets A / B$
\end{description}

\subsection{Konstanten}
\label{subsection:Dokumentation-Implementierung-Hardwareerweiterung-Konstanten}

\begin{itemize}
    \item $(e0000000)_{16}$
    \item $(2^{31})_{10}$
    \item $(2^{28})_{10}$
    \item $(2^{16})_{10}$
    \item $(2^5)_{10}$
    \item $(2)_{10}$
\end{itemize}

\subsection{Konfigurationsdatei}
\label{subsection:Dokumentation-Implementierung-Hardwareerweiterung-Konfigurationsdatei}

Mit Hilfe der bereitgestellten Beispiele konnte eine Erweiterungsdatei für die die oben genannten Erweiterungen erstellt werden. Diese kann in den Simulator geladen werden (siehe \autoref{chapter:Dokumentation-Simulation}).

Die Erweiterungskonfiguration befindet sich im Anhang in \autoref{chapter:Anhang-Konfiguration}.

\section{Algorithmus}
\label{section:Dokumentation-Implementierung-Algorithmus}

\autoref{chapter:Anhang-RtNotation}

\section{Steuertabelle}
\label{section:Dokumentation-Implementierung-Steuertabelle}

Siehe \autoref{chapter:Anhang-Steuertabelle}. Noch erweitern, Kommentar schreiben, ein bisschen erklären usw.
