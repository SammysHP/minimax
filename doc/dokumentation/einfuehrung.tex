\chapter{Einführung}
\label{chapter:Dokumentation-Einfuehrung}

Im Folgenden wird der Algorithmus aus \autoref{subsection:Pflichtenheft-SystemtechnischeLoesung-Loesungsansatz-Ansatz2} von \autoref{part:Pflichtenheft} konkretisiert und implementiert.

Dazu wird die Konfigurationsdatei für die nötige Hardwareerweiterung erstellt. Parallel dazu wird der Algorithmus auf Basis des Flussdiagramms zuerst in Pseudocode verfasst, welcher danach in die RT"=Notation übertragen wird. Diese RT"=Notation wird anschließend in eine Steuertabelle für die Minimax"=Maschine übersetzt.\\
Eine genauere Beschreibung inklusive Ergebnisse befindet sich in \autoref{chapter:Dokumentation-Implementierung}, der Code im Anhang (\autoref{part:Anhang}).

Mit diesen Vorbereitungen kann der Simulator gestartet werden. Eine Erläuterung befindet sich in \autoref{chapter:Dokumentation-Simulation}.

Abschließend wird das Ergebnis in \autoref{chapter:Dokumentation-BenchmarkBewertung} analysiert und bewertet.


\section{Protokollierung}
\label{section:Dokumentation-Einfuehrung-Protokollierung}

\begin{tabularx}{\textwidth}{|l|l|X|c|}
    \hline
    Datum & Bearbeiter & Vorgang & Zeit \\
    \hline
    \hline
    10.01.2012 & Wildemann & Grundentwurf für Algorithmus, Pseudocode, Abstraktion, Diagramm & 2,5 h \\
    \hline
    16.01.2012 & Greiner & Grundgerüst für Dokumentation & 1 h \\
    \hline
    18.01.2012 & Entrup, Wildemann & Simulator analysiert, Workarounds für Probleme gesucht & 3 h \\
    \hline
    19.01.2012 & Entrup, Wildemann & RT"=Notation & 2 h\\
    \hline
    20.01.2012 & Matthaei, Wildemann & Steuertabelle & 3 h\\
    \hline
    22.01.2012 & alle & Tests und Korrekturen & 10 h\\
    \hline
    23.01.2012 & Wildemann & Optimierungen & 1,5 h \\
               & alle & Tests, Korrekturen, Benchmarkberechnung, Aufräumen & 5 h \\
               & Greiner & Aktualisierung der Dokumentation & 1,5 h \\
    \hline
\end{tabularx}
