\chapter{Einführung}
\label{chapter:Dokumentation-Einfuehrung}

Im Folgenden wird der Algorithmus aus \autoref{subsection:Pflichtenheft-SystemtechnischeLoesung-Loesungsansatz-Ansatz2} von \autoref{part:Pflichtenheft} konkretisiert und implementiert.

Dazu wird die Konfigurationsdatei für die nötige Hardwareerweiterung erstellt. Parallel dazu wird der Algorithmus auf Basis des Flussdiagramms zuerst in Pseudocode verfasst, welcher danach in die RT"=Notation übertragen wird. Diese RT"=Notation wird anschließend in eine Steuertabelle für die Minimax"=Maschine übersetzt.\\
Eine genauere Beschreibung inklusive Ergebnisse befindet sich in \autoref{chapter:Dokumentation-Implementierung}, der Code im Anhang (\autoref{part:Anhang}).

Mit diesen Vorbereitungen kann der Simulator gestartet werden. Eine Erläuterung befindet sich in \autoref{chapter:Dokumentation-Simulation}.

Abschließend wird das Ergebnis in \autoref{chapter:Dokumentation-BenchmarkBewertung} analysiert und bewertet.
