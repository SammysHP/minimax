\chapter{Einführung}
\label{chapter:Pflichtenheft-Einfuehrung}

\section{Zusammenfassung}
\label{section:Pflichtenheft-Einfuehrung-Zusammenfassung}

In diesem Projekt soll für eine bestehende Minimax-Implementierung ein Algorithmus (siehe \autoref{chapter:Pflichtenheft-Sollzustand}) entwickelt und ggf. die vorhandene Hardware um neue Komponenten erweitert werden, die für die Algorithmenimplementierung notwendig sind.

\section{Projektumfeld}
\label{section:Pflichtenheft-Einfuehrung-Projektumfeld}

Dieses Projekt findet im Rahmen der Veranstaltung "`Hardwarepraktikum"' an der Leibniz Universität Hannover\footnote{\url{http://www.uni-hannover.de/}} statt. Auftraggeber ist das SRA\footnote{Institut für Systems Engineering -- System- und Rechnerarchitektur Appelstr. 4, 30167 Hannover}, vertreten durch Yaser Chaaban\footnote{Yaser Chaaban, System- und Rechnerarchitektur, Appelstr. 4 (1.Stock), 30167 Hannover, Tel.: (0511) 762-19729, \url{mailto:chaaban@sra.uni-hannover.de}}.

\subsection{Ablauf}
\label{subsection:Pflichtenheft-Einfuehrung-Projektumfeld-Ablauf}

Der Zeitplan sieht vor, das Pflichtenheft zum 11.01.2012 vorzulegen (verbunden mit einem Projekttreffen), sowie die Abgabe der Lösung inklusive Dokumentation bis zum 24.01.2012 um 11 Uhr. Am 25.02.2012 findet anschließend ab 9 Uhr der Abschlussvortrag im SRA statt. Für eine Gruppe ist ein öffentlicher Vortrag geplant, sofern dies gewünscht wird und die nötige Zeit vorhanden ist.

Nach Abschluss wird dieses Projekt mit einem "`bestanden"' oder "`nicht bestanden"' bewertet. Für ein "`bestanden"' sind eine ausreichende Bearbeitung der Aufgaben sowie die Präsentation notwendig.

\subsection{Personal}
\label{subsection:Pflichtenheft-Einfuehrung-Projektumfeld-Personal}

Diese Gruppe besteht während der gesamten Durchführung (sofern keine Ereignisse außerhalb unseres Einflussbereiches auftreten)  aus folgenden Teilnehmern:

\begin{multicols}{2}
    \parbox{\textwidth}{
        Gerion Entrup\\
        Matrikelnr.: xxx\\
        \href{mailto:xxx@yyy.zz}{xxx@yyy.zz}
    }

    \parbox{\textwidth}{
        Martin Mattheis\\
        Matrikelnr.: xxx\\
        \href{mailto:xxx@yyy.zz}{xxx@yyy.zz}
    }

    \parbox{\textwidth}{
        Sergej Wildemann\\
        Matrikelnr.: xxx\\
        \href{mailto:xxx@yyy.zz}{xxx@yyy.zz}
    }

    \parbox{\textwidth}{
        Sven Karsten Greiner\\
        Matrikelnr.: xxx\\
        \href{mailto:xxx@yyy.zz}{xxx@yyy.zz}
    }
\end{multicols}

Als Vertreter und Ansprechpartner wurde Gerion Entrup gewählt.

