\chapter{Istzustand}
\label{chapter:Pflichtenheft-Istzustand}

\section{Der Simulator}
\label{chapter:Pflichtenheft-Istzustand-Simulator}

Vom Auftraggeber ist ein lauffähiger Simulator einer Minimax-Maschine gegeben. Dieser wird über Textdateien gesteuert, welche die Hardwarekomponenten in einer speziellen Beschreibungssprache definieren. Dazu zählen unter anderem Register, Multiplexer, ALU, CU, Sign, Speicher und Verbindungen. Durch hinzufügen weiterer Steuerungsdateien lässt sich die virtuelle Maschine erweitern.

Der Simulator selbst ist in Java (ab 1.3) verfasst und somit plattformunabhängig. Ein Quellcode liegt nicht vor.

Begleitend ist ein umfangreiches Handbuch in PDF"=Form vorhanden \cite{minimax-handbuch}, welches alle wichtigen Eigenschaften des Simulators und der Basiskonfiguration erläutert. Weitere Informationen zum Simulator können diesem entnommen werden.


\section{Projektdateien}
\label{chapter:Pflichtenheft-Istzustand-Projektdateien}

In Bezug auf die Aufgabe sind außerdem Speicherabbilder für einen einheitlichen Benchmark"=Vergleich gegeben. Das Format kann vom Simulator ohne Anpassungen gelesen werden.

Weiterhin wurden die Aufgaben und Anforderungen als PDF"=Dokumente zur Verfügung gestellt. Einige Punkte daraus werden bei Gelegenheit in diesem Dokument aufgeführt und erläutert.
