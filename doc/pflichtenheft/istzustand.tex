\chapter{Istzustand}
\label{chapter:Pflichtenheft-Istzustand}

\section{Der Simulator}
\label{chapter:Pflichtenheft-Istzustand-Simulator}

Vom Auftraggeber ist ein lauffähiger Simulator einer Minimax-Maschine gegeben. Dieser wird über Textdateien gesteuert, welche die Hardwarekomponenten in einer speziellen Beschreibungssprache definieren. Dazu zählen unter anderem Register, Multiplexer, ALU, CU, Sign, Speicher und Verbindungen. Durch hinzufügen weiterer Steuerungsdateien lässt sich die virtuelle Maschine erweitern.

Der Simulator selbst ist in Java (ab 1.3) realisiert und somit plattformunabhängig. Ein Quellcode liegt nicht vor.

Begleitend ist ein umfangreiches Handbuch in PDF"=Form vorhanden (siehe \cite{minimax-handbuch}), welches alle wichtigen Eigenschaften des Simulators und der Basiskonfiguration erläutert. Weitere Informationen zum Simulator können diesem entnommen werden. Für die Bedienung notwendige Schritte werden später im \hyperref[part:Dokumentation]{zweiten Teil} dieser Dokumentation erläutert.


\section{Projektdateien}
\label{chapter:Pflichtenheft-Istzustand-Projektdateien}

Die Aufgaben und Anforderungen wurden als PDF"=Dokument (siehe \cite{aufgabenblatt}) zur Verfügung gestellt. Einige Punkte daraus werden bei Gelegenheit in diesem Dokument aufgeführt und erläutert.

In Bezug auf die Aufgabe sind außerdem Speicherabbilder für einen einheitlichen Benchmark"=Vergleich gegeben. Das Format kann vom Simulator ohne Anpassungen gelesen werden.

